\chapter{Tie Strength} % (fold)
\label{cha:tie_strength}
    \section{Critical threshold} % (fold)
    \label{sec:critical_threshold}
        As described in chapter 8 of \cite{network_science}, we have obtained the \textbf{critical threshold}
        representing the fraction of the nodes that have to be removed to break apart out network. This fraction,
        represented by $f_c$, is obtained by the following formula:

        \begin{equation*}
            f_c = 1 - \frac{1}
            {\frac{\gamma - 2}{3 - \gamma}k^{\gamma - 2}_{\mathit{min}}k^{3 - \gamma}_{\mathit{max}} \ - \  1} =
            1 - \frac{1}{1.50 * 1^{0.6} * 19073^{0.4} \ - \ 1} = 0.99
        \end{equation*}

    which, remembering that the $\gamma$ for our scale-free network corresponds to $2.6$ and that $k_{\mathit{min}}$
    and $k_{\mathit{max}}$ are eguals to $1$ and $19073$ respectively, tells us that, in order to break apart our
    network it is mandatory to remove the $99\%$ of the nodes. Keeping in mind that our network is, in fact, a finite
    network, we can adjust the obtained result by utilizing the following formula, still described in chapter 8 of
    \cite{network_science}:

    \begin{equation*}
        f_c \approx 1 - \frac{C}{N^{\frac{3 - \gamma}{\gamma - 1}}} \approx 1.00
    \end{equation*}

    where $C = \frac{1}{\sum_{k=1}^{\infty}k^{-\gamma}}$ is a constant and $N$ represents the number of nodes of the
    network. As we can see, this new approximation tells us that in order to break apart our network the totality of
    its nodes has to be removed.
    % section critical_threshold (end)
% chapter tie_strength (end)
