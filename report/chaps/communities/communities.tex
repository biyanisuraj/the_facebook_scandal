\chapter{Communities discovery} % (fold)
\label{cha:communities_discovery}

In this chapter we'll provide the results obtained by applying \textbf{K-Clique}, \textbf{Label Propagation},
\textbf{Louvain}, \textbf{Girvan-Newman} and \textbf{Demon} to a sample of $1000$ nodes taken from the original
network. We've chosen to sample the crawled data in order to ease the application of the various algorithms.
Each partition is evaluated by applying an implementation of the scoring functions listed in
\cite{comm_ground_truth}, and, for each algorithm, the results are represented in a table. Together with the
results of the scoring functions are also provided the total number of communities discoverd (Communities), the
number of nodes in the smallest/biggest community (Smallest/Biggest), the hashtags utilized in the biggest
community (Tags) and finally the languages of the users in the biggest community (Langs).

\section{K-Clique} % (fold)
\label{sec:k_clique}
    We have chosen to apply the \textbf{K-Clique} algorithm, described in \cite{k_clique}, to the sample using
    three different values for $k$: $3$, $4$ and $5$, respectively. The results are represented in Table
    \ref{kclique}. As we can see, even if the result is not so good, the best partition is obtained by using $k$
    eguals to $3$, which returns a low modularity partition composed by low degree nodes. It is interesting to
    note that all the biggest partitions returned by the various applications of the algorithm are composed only
    by english speaking users.

    \begin{table}[H]
        \centering
        \begin{subtable}{\textwidth}
            \resizebox{\textwidth}{!}{
                \begin{tabular}{| c | c | c | c | c | c | c | c | c | c |}
                    \hline
                    K & Communities & Biggest & Smallest & Modularity & Conductance & IED & AND &  Tags &
                    Langs \\
                    \hline
                    3 & 29 & 51 & 3 & 0.14 & 0.62 & 0.20 & 2.72 & zuckerberg,
                    cambridgeanalytica, deletefacebook, facebook, facebookgate & English, Deutsch \\
                    \hline
                    4 & 13 & 14 & 4 & 0.020 & 0.68 & 0.21 & 4.23 & cambridgeanalytica, deletefacebook, facebook
                    & English \\
                    \hline
                    5 & 5 & 12 & 5 & 0.011 & 0.63 & 0.23 & 4.83 & cambridgeanalytica, deletefacebook, facebook
                    & English \\
                    \hline
                \end{tabular}
            }
        \end{subtable}
        \caption{Evaluation of the partitions obtained by the application of the K-Clique algorithm.}
        \label{kclique}
    \end{table}

% section k_clique (end)

\section{Label Propagation} % (fold)
\label{sec:label_propagation}
    In Table \ref{labelprop} are represented the results of the application of the \textbf{Label Propagation}
    algorithm, described in \cite{label_propagation}. According to the modularity score the partition provided by
    this algorithm represent a good subdivision of the original network, even if it is composed for the vast
    majority by small communities, as suggested by the high number of communities and the low value for the
    Average Node Degree score. As for the kclique algorithm, it is interesting to note that the biggest community
    is composed only by english speaking users.

    \begin{table}[H]
        \centering
        \begin{subtable}{\textwidth}
            \resizebox{\textwidth}{!}{
                \begin{tabular}{| c | c | c | c | c | c | c | c | c |}
                    \hline
                    Communities & Biggest & Smallest & Modularity & Conductance & IED & AND &  Tags & Langs \\
                    \hline
                    1278 & 136 & 1 & 0.68 & 0.28 & 0.19 & 1.38 & cambridgeanalytica,
                    deletefacebook, privacy, zuckerberg, facebook, facebookgate & Various  \\
                    \hline
                \end{tabular}
            }
        \end{subtable}
        \caption{Evaluation of the partition obtained by the application of the Label Propagaion algorithm.}
        \label{labelprop}
    \end{table}

% section label_propagation (end)

\section{Louvain} % (fold)
\label{sec:louvain}
    The application of the \textbf{Louvain}, described in \cite{louvain}, along with the last iteration of the
    Girvan-Newman algorithm, returns the best partition among all the partitions returned by the other algorithms.
    The results of its application are represented in Table \ref{louvain}. As for the Label Propagation algorithm,
    this partition also is composed by an high number of small communities, composed by nodes with degree betweem
    $1$ and $2$.

    \begin{table}[H]
        \centering
        \begin{subtable}{\textwidth}
            \resizebox{\textwidth}{!}{
                \begin{tabular}{| c | c | c | c | c | c | c | c | c |}
                    \hline
                    Communities & Biggest & Smallest & Modularity & Conductance & IED & AND &  Tags & Langs \\
                    \hline
                    1193 & 133 & 1 & 0.76 & 0.042 & 0.17 & 1.60 & cambridgeanalytica,
                    deletefacebook, privacy, zuckerberg, facebook, facebookgate & Various \\
                    \hline
                \end{tabular}
            }
        \end{subtable}
        \caption{Evaluation of the partition obtained by the application of the Louvain algorithm.}
        \label{louvain}
    \end{table}

% section louvain (end)

\section{Girvan-Newman} % (fold)
\label{sec:girvan_newman}
    For the \textbf{Girvan-Newman} algorithm, described in \cite{girvan_newman}, we've decided to record
    the results of $5$ iterations over the sample network. The results are represented in Table
    \ref{girvan_newman}. As you can see, the first iteration returns a very poor partition, with a low modularity
    score, due to the fact that the edge with the highest betweenness centrality in the starting sample network
    doesn't provide a good grade of separation among the nodes of the network. With the second iteration, and the
    ones after that, there is a consistent improvement either in the modularity score and in the other measures.
    \begin{table}[H]
        \centering
        \begin{subtable}{\textwidth}
            \resizebox{\textwidth}{!}{
                \begin{tabular}{| c | c | c | c | c | c | c | c | c | c |}
                    \hline
                    Iteration & Communities & Biggest & Smallest & Modularity & Conductance & IED & AND &  Tags &
                    Langs \\
                    \hline
                    1 & 1181 & 703 & 1 & 0.17 & 0.00098 & 0.22 & 1.23 & cambridgeanalytica,
                    deletefacebook, privacy, zuckerberg, facebook, facebookgate & Various \\
                    \hline
                    2 & 1182 & 659 & 1 & 0.24 & 0.0019 & 0.21 & 1.27 & cambridgeanalytica,
                    deletefacebook, privacy, zuckerberg, facebook, facebookgate & Various \\
                    \hline
                    3 & 1183 & 588 & 1 & 0.38 & 0.0048 & 0.21 & 1.35 & cambridgeanalytica, deletefacebook, privacy,
                    zuckerberg, facebook, facebookgate & Various \\
                    \hline
                    4 & 1184 & 575 & 1 & 0.40 & 0.0081 & 0.20 & 1.34 & cambridgeanalytica, deletefacebook, privacy,
                    zuckerberg, facebook, facebookgate & Various \\
                    \hline
                    5 & 1185 & 458 & 1 & 0.57 & 0.011 & 0.20 & 1.40 & zuckerberg, cambridgeanalytica, deletefacebook,
                    facebook, facebookgate & Various \\
                    \hline
                \end{tabular}
            }
        \end{subtable}
        \caption{Evaluation of the partition obtained by the application of the Girvan-Newman algorithm.}
        \label{girvan_newman}
    \end{table}

    In general, the partitions returned by the five iterations of the algorithms are all composed by an high number
    of small communities.

% section girvan_newman (end)

\section{Demon} % (fold)
\label{sec:demon}
    Finally in Table \ref{demon} we provide the results of the application of the \textbf{Demon} algorithm,
    described in \cite{demon}, that we tested for three different values of $\epsilon$, $0.25$, $0.50$ and $0.75$,
    respectively. In general, the three partitions are not so good from the point of view of the modularity score,
    with the first application of the algorithm beign the best among the three
    Contrary to the results obtained by the application of the other algorithms, the partitions for the Demon
    algorithm are all composed by a small number of communities, with the last one being the "denser" one, which
    are composed by nodes with a degree between $3$ and $4$.

    \begin{table}[H]
        \centering
        \begin{subtable}{\textwidth}
            \resizebox{\textwidth}{!}{
                \begin{tabular}{| c | c | c | c | c | c | c | c | c | c |}
                \hline
                Epsilon & Communities & Biggest & Smallest & Modularity & Conductance & IED & AND &  Tags & Langs \\
                \hline
                0.10 & 10 & 147 & 4 & 0.07 & 0.46 & 0.082 & 4.41 &
                zuckerberg, cambridgeanalytica, deletefacebook, facebook, facebookgate & Various \\
                \hline
                0.25 & 11 & 63 & 4 & 0.095 & 0.44 & 0.094 & 4.31 &
                zuckerberg, cambridgeanalytica, deletefacebook, facebook, facebookgate & English, Deutsch \\
                \hline
                0.50 & 21 & 43 & 4 & 0.11 & 0.57 & 0.10 & 4.15 &
                zuckerberg, cambridgeanalytica, deletefacebook, facebook, facebookgate & English, Deutsch \\
                \hline
                0.75 & 39 & 25 & 4 & 0.62 & 0.51 & 0.12 & 3.96 &
                zuckerberg, cambridgeanalytica, deletefacebook, facebook, facebookgate & English, Deutsch \\
                \hline
                0.90 & 89 & 24 & 4 & 0.071 & 0.68 & 0.15 & 3.65 &
                zuckerberg, cambridgeanalytica, deletefacebook, facebook, facebookgate & English, Deutsch \\
                \hline
                \end{tabular}
            }
        \end{subtable}
        \caption{Evaluation of the partition obtained by the application of the Demon algorithm.}
        \label{demon}
    \end{table}

% section demon (end)

\section{Comparisons} % (fold)
\label{sec:comparisons}
    In this final section of the chapter we compare the algorithms used so far by confronting the best instances
    among the iterations provided in the past sections. The comparisons are performed by using the NF$1$ score, as
    described in \cite{rossetti2016}.

    \begin{table}[H]
        \centering
        \begin{subtable}{\textwidth}
            \resizebox{\textwidth}{!}{
                \begin{tabular}{| c | c | c | c | c | c | c | c | c |}
                    \hline
                    A1 & A2 & F1 mean & Ground Truth Communities & Identified Communities & Community Ratio &
                    Ground Truth Matched & Node Coverage & NF1 \\
                    \hline
                    K-Clique 3 & Label Propagation & 0.44 & 714.0 & 21.0 & 0.029 & 0.022 & 0.10 & 0.0076 \\
                    \hline
                    K-Clique 3 & Louvain & 0.32 & 683.0 & 21.0 & 0.031 & 0.016 & 0.10 & 0.0027 \\
                    \hline
                    K-Clique 3 & Girvan-Newman 5 & 0.25 & 677.0 & 21.0 & 0.031 & 0.012 & 0.10 & 0.0011 \\
                    \hline
                    K-Clique 3 & Demon 0.25 & 0.92 & 6.0 & 21.0 & 3.5 & 1.0 & 1.44 & 0.24 \\
                    \hline
                    Label Propagation & Louvain & 0.96 & 683.0 & 714.0 & 1.05 & 1.0 & 1.0 & 0.92 \\
                    \hline
                    Label Propagation & Girvan-Newman 5 & 0.95 & 677.0 & 714.0 & 1.06 & 1.0 & 1.0
                    & 0.90 \\
                    \hline
                    Label Propagation & Demon 0.25 & 0.38 & 6.0 & 714.0 & 119.0 & 1.0 & 14.17
                    & 0.0032 \\
                    \hline
                    Louvain & Girvan-Newman 5 & 0.99 & 677.0 & 683.0 & 1.01 & 1.0 & 1.0 & 0.98 \\
                    \hline
                    Louvain & Demon 0.25 & 0.38 & 6.0 & 683.0 & 113.83 & 1.0 & 14.17 & 0.0033 \\
                    \hline
                    Girvan-Newman 5 & Demon 0.25 & 0.36 & 6.0 & 677.0 & 112.83 & 1.0 & 14.17 & 0.0032 \\
                    \hline
                \end{tabular}
            }
        \end{subtable}
        \caption{Comparisons among the best iterations of the algorithms utilized in this chapter.}
        \label{comparisons}
    \end{table}

    In Table \ref{comparisons} we can see the comparisons among the best iterations of the algorithms utilized
    during the community discovery phase. As we can see, the best results are returned by the comparisons between
    the Label Propagation, Louvain and Girvan-Newman (fifth iteration) algorithms, since that, as we've seen in
    the past sections, they are in fact the best performing algorithms among the ones we've tested for the
    communities discovery.
% section comparisons (end)

% chapter communities_discovery (end)
