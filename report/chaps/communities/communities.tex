\chapter{Communities discovery} % (fold)
\label{cha:communities_discovery}

In this chapter we'll provide the results obtained by applying \textbf{K-Clique}, \textbf{Label Propagation},
\textbf{Louvain}, \textbf{Girvan-Newman} and \textbf{Demon} to a sample of $2000$ nodes taken from the original
network. We've chosen to sample the crawled data in order to ease the application of the various algorithms.
Each partition is evaluated by applying an implementation of the scoring functions listed in
\cite{comm_ground_truth}, and, for each algorithm, the results are represented in a table. Together with the
results of the scoring functions are also provided the total number of communities discoverd (Communities), the
number of nodes in the smallest/biggest community (Smallest/Biggest), the hashtags utilized in the biggest
community (Tags) and finally the languages of the users in the biggest community (Langs). Finally it is also
provided an evaluation for the subset of the original network composed only by italian users.

\section{K-Clique} % (fold)
\label{sec:k_clique}
    We have chosen to apply the \textbf{K-Clique} algorithm, described in \cite{k_clique}, to the sample using
    three different values for $k$: $3$, $4$ and $5$, respectively. The results are represented in Table
    \ref{kclique}. As we can see, even if the result is not so good, the best partition is obtained by using $k$
    eguals to $3$, which returns a low modularity partition composed by low degree nodes. It is interesting to
    note that the biggest partitions returned by the first application of the algorithm is composed
    by english and deutsch speaking users, while the other iterations returns community composed only by english
    speaking users.

    \begin{table}[H]
        \centering
        \begin{subtable}{\textwidth}
            \resizebox{\textwidth}{!}{
                \begin{tabular}{| c | c | c | c | c | c | c | c | c | c |}
                    \hline
                    K & Communities & Biggest & Smallest & Modularity & Conductance & IED & AND &  Tags &
                    Langs \\
                    \hline
                    3 & 29 & 51 & 3 & 0.14 & 0.62 & 0.20 & 2.72 & zuckerberg,
                    cambridgeanalytica, deletefacebook, facebook, facebookgate & English, Deutsch \\
                    \hline
                    4 & 13 & 14 & 4 & 0.020 & 0.68 & 0.21 & 4.23 & cambridgeanalytica, deletefacebook, facebook
                    & English \\
                    \hline
                    5 & 5 & 12 & 5 & 0.011 & 0.63 & 0.23 & 4.83 & cambridgeanalytica, deletefacebook, facebook
                    & English \\
                    \hline
                \end{tabular}
            }
        \end{subtable}
        \caption{Evaluation of the partitions obtained by the application of the K-Clique algorithm.}
        \label{kclique}
    \end{table}

% section k_clique (end)

\section{Label Propagation} % (fold)
\label{sec:label_propagation}
    In Table \ref{labelprop} are represented the results of the application of the \textbf{Label Propagation}
    algorithm, described in \cite{label_propagation}. According to the modularity score the partition provided by
    this algorithm represent a good subdivision of the original network, even if it is composed for the vast
    majority by small communities, as suggested by the high number of communities and the low value for the
    Average Node Degree score. Since this partition is composed by an high number of communities, the biggest
    community aggregates a vast variety of languages.

    \begin{table}[H]
        \centering
        \begin{subtable}{\textwidth}
            \resizebox{\textwidth}{!}{
                \begin{tabular}{| c | c | c | c | c | c | c | c | c |}
                    \hline
                    Communities & Biggest & Smallest & Modularity & Conductance & IED & AND &  Tags & Langs \\
                    \hline
                    1278 & 136 & 1 & 0.68 & 0.28 & 0.19 & 1.38 & cambridgeanalytica,
                    deletefacebook, privacy, zuckerberg, facebook, facebookgate & Various  \\
                    \hline
                \end{tabular}
            }
        \end{subtable}
        \caption{Evaluation of the partition obtained by the application of the Label Propagaion algorithm.}
        \label{labelprop}
    \end{table}

% section label_propagation (end)

\section{Louvain} % (fold)
\label{sec:louvain}
    The application of the \textbf{Louvain} algorithm, described in \cite{louvain}, along with the iteration
    of the Label Propagation algorithm, returns the best partition among all the partitions returned by the other
    algorithms. The results of its application are represented in Table \ref{louvain}. As for the Label Propagation
    algorithm, this partition also is composed by an high number of small communities, and the biggest community is composed by users who speaks various languages.

    \begin{table}[H]
        \centering
        \begin{subtable}{\textwidth}
            \resizebox{\textwidth}{!}{
                \begin{tabular}{| c | c | c | c | c | c | c | c | c |}
                    \hline
                    Communities & Biggest & Smallest & Modularity & Conductance & IED & AND &  Tags & Langs \\
                    \hline
                    1193 & 133 & 1 & 0.76 & 0.042 & 0.17 & 1.60 & cambridgeanalytica,
                    deletefacebook, privacy, zuckerberg, facebook, facebookgate & Various \\
                    \hline
                \end{tabular}
            }
        \end{subtable}
        \caption{Evaluation of the partition obtained by the application of the Louvain algorithm.}
        \label{louvain}
    \end{table}

% section louvain (end)

\section{Girvan-Newman} % (fold)
\label{sec:girvan_newman}
    For the \textbf{Girvan-Newman} algorithm, described in \cite{girvan_newman}, we've decided to record
    the results of $5$ iterations over the sample network. The results are represented in Table
    \ref{girvan_newman}. As you can see, the first three iterations returns very poor partitions, with low
    modularity scores, due to the fact that the edges with the highest betweenness centrality selected in the various
    iterations doesn't provide a good grade of separation among the nodes of the network. With the fourth and fifth
    iterations, there is a consistent improvement either in the modularity score and in the other measures.
    \begin{table}[H]
        \centering
        \begin{subtable}{\textwidth}
            \resizebox{\textwidth}{!}{
                \begin{tabular}{| c | c | c | c | c | c | c | c | c | c |}
                    \hline
                    Iteration & Communities & Biggest & Smallest & Modularity & Conductance & IED & AND &  Tags &
                    Langs \\
                    \hline
                    1 & 1181 & 703 & 1 & 0.17 & 0.00098 & 0.22 & 1.23 & cambridgeanalytica,
                    deletefacebook, privacy, zuckerberg, facebook, facebookgate & Various \\
                    \hline
                    2 & 1182 & 659 & 1 & 0.24 & 0.0019 & 0.21 & 1.27 & cambridgeanalytica,
                    deletefacebook, privacy, zuckerberg, facebook, facebookgate & Various \\
                    \hline
                    3 & 1183 & 588 & 1 & 0.38 & 0.0048 & 0.21 & 1.35 & cambridgeanalytica, deletefacebook, privacy,
                    zuckerberg, facebook, facebookgate & Various \\
                    \hline
                    4 & 1184 & 575 & 1 & 0.40 & 0.0081 & 0.20 & 1.34 & cambridgeanalytica, deletefacebook, privacy,
                    zuckerberg, facebook, facebookgate & Various \\
                    \hline
                    5 & 1185 & 458 & 1 & 0.57 & 0.011 & 0.20 & 1.40 & zuckerberg, cambridgeanalytica, deletefacebook,
                    facebook, facebookgate & Various \\
                    \hline
                \end{tabular}
            }
        \end{subtable}
        \caption{Evaluation of the partition obtained by the application of the Girvan-Newman algorithm.}
        \label{girvan_newman}
    \end{table}

    In general, the partitions returned by the five iterations of the algorithms are all composed by an high number
    of small communities, with the biggest community growing smaller iteration after iteration. This kind of
    fragmentation, as seen before, for every iteration produces a biggest community with diffent types of users.

% section girvan_newman (end)

\section{Demon} % (fold)
\label{sec:demon}
    Finally in Table \ref{demon} we provide the results of the application of the \textbf{Demon} algorithm,
    described in \cite{demon}, that we tested for five different values of $\epsilon$, $0.10$, $0.25$, $0.50$,
    $0.75$ and $0.90$, respectively. In general, the five partitions are not so good from the point of view of the
    modularity score, with the third application of the algorithm beign the best.
    Contrary to the results obtained by the application of the other algorithms, the partitions for the Demon
    algorithm are all composed by a small number of communities.

    \begin{table}[H]
        \centering
        \begin{subtable}{\textwidth}
            \resizebox{\textwidth}{!}{
                \begin{tabular}{| c | c | c | c | c | c | c | c | c | c |}
                \hline
                Epsilon & Communities & Biggest & Smallest & Modularity & Conductance & IED & AND &  Tags & Langs \\
                \hline
                0.10 & 10 & 147 & 4 & 0.07 & 0.46 & 0.082 & 4.41 &
                zuckerberg, cambridgeanalytica, deletefacebook, facebook, facebookgate & Various \\
                \hline
                0.25 & 11 & 63 & 4 & 0.095 & 0.44 & 0.094 & 4.31 &
                zuckerberg, cambridgeanalytica, deletefacebook, facebook, facebookgate & English, Deutsch \\
                \hline
                0.50 & 21 & 43 & 4 & 0.11 & 0.57 & 0.10 & 4.15 &
                zuckerberg, cambridgeanalytica, deletefacebook, facebook, facebookgate & English, Deutsch \\
                \hline
                0.75 & 39 & 25 & 4 & 0.068 & 0.62 & 0.12 & 3.96 &
                zuckerberg, cambridgeanalytica, deletefacebook, facebook, facebookgate & English, Deutsch \\
                \hline
                0.90 & 89 & 24 & 4 & 0.071 & 0.68 & 0.15 & 3.65 &
                zuckerberg, cambridgeanalytica, deletefacebook, facebook, facebookgate & English, Deutsch \\
                \hline
                \end{tabular}
            }
        \end{subtable}
        \caption{Evaluation of the partition obtained by the application of the Demon algorithm.}
        \label{demon}
    \end{table}

% section demon (end)

\section{Comparisons} % (fold)
\label{sec:comparisons}
    In this section of the chapter we compare the algorithms used so far by confronting the best instances
    among the iterations provided in the past sections. The comparisons are performed by using the NF$1$ score, as
    described in \cite{rossetti2016}.

    \begin{table}[H]
        \centering
        \begin{subtable}{\textwidth}
            \resizebox{\textwidth}{!}{
                \begin{tabular}{| c | c | c | c | c | c | c | c | c |}
                    \hline
                    A1 & A2 & F1 mean & Ground Truth Communities & Identified Communities & Community Ratio &
                    Ground Truth Matched & Node Coverage & NF1 \\
                    \hline
                    K-Clique 3 & Label Propagation & 0.44 & 714.0 & 21.0 & 0.029 & 0.022 & 0.10 & 0.0076 \\
                    \hline
                    K-Clique 3 & Louvain & 0.32 & 683.0 & 21.0 & 0.031 & 0.016 & 0.10 & 0.0027 \\
                    \hline
                    K-Clique 3 & Girvan-Newman 5 & 0.25 & 677.0 & 21.0 & 0.031 & 0.012 & 0.10 & 0.0011 \\
                    \hline
                    K-Clique 3 & Demon 0.50 & 0.94 & 9.0 & 21.0 & 2.33 & 0.78 & 1.44 & 0.24 \\
                    \hline
                    Label Propagation & Louvain & 0.94 & 1193.0 & 1278.0 & 1.07 & 1.0
                    & 1.0 & 0.87 \\
                    \hline
                    Label Propagation & Girvan-Newman 5 & 0.93 & 1185.0 & 1278.0 & 1.08 & 1.0 & 1.0 &
                    0.86 \\
                    \hline
                    Label Propagation & Demon 0.50 & 0.26 & 21.0 & 1278.0 & 60.86 & 0.76 & 10.24 &
                    0.0025 \\
                    \hline
                    Louvain & Girvan-Newman 5 & 0.99 & 1185.0 & 1193.0 & 1.0068 & 1.0 & 1.0 & 0.99 \\
                    \hline
                    Louvain & Demon 0.50 & 0.32 & 21.0 & 1193.0 & 56.81 & 0.62 & 10.24 & 0.0021 \\
                    \hline
                    Girvan-Newman 5 & Demon 0.50 & 0.36 & 21.0 & 1185.0 & 56.43 & 0.24 & 10.24 & 0.00036 \\
                    \hline
                \end{tabular}
            }
        \end{subtable}
        \caption{Comparisons among the best iterations of the algorithms utilized in this chapter.}
        \label{comparisons}
    \end{table}

    In Table \ref{comparisons} we can see the comparisons among the best iterations of the algorithms
    utilized during the community discovery phase. As we can see by the results, the best comparisons are
    the ones between K-Clique and Demon , with, respectively, $k = 3$ and $\epsilon = 0.50$, between Label
    Propagation and Louvain/Girvan-Newman (fifth iteration) and finally between Louvain and Girvan-Newman
    (fifth iteration).
% section comparisons (end)

\section{The italian subgraph} % (fold)
\label{sec:the_italian_subgraph}
    In order to better understand the composition of our network, we selected among the nodes only the ones
    composed by \textbf{italian users}, and on this subgraph we've applied the same algorithms described
    in the past sections. The obtained partitions make clear to us that the italian users inside our
    network aren't disposed into a community-like shape. The best obtained result is the one returned by the
    Louvain algorithm, which return a modularity score of $0.26$ and a partition composed by $6$
    communities, the biggest and the smallest being composed by $817$ and $2$ users, respectively. Being
    the partition composed by only $6$ communities, the average node degree is bigger
    ($\approx 17$) that the ones returned by the other evaluations.
% section the_italian_subgraph (end)

% chapter communities_discovery (end)
