In this chapter we'll provide the results obtained by applying \textbf{K-Clique}, \textbf{Label Propagation},
\textbf{Louvain}, \textbf{Girvan-Newman} and \textbf{Demon} to a sample of $1000$ nodes taken from the original
network. We've chosen to sample the crawled data in order to ease the application of the various algorithms.
For each application of the algorithms we provide a table containing the measures for evaluating the obtained
partitions (Modularity, Conductance, IED and AND), the total number of communities discoverd
(Communities), the number of nodes in the smallest/biggest community (Smallest/Biggest), the hashtags utilized
in the biggest community (Tags) and finally the languages of the users in the biggest community (Langs).

\section{K-Clique} % (fold)
\label{sec:k_clique}
    We have chosen to apply the \textbf{K-Clique} algorithm to the sample using three different values for
    $k$: $3$, $4$ and $5$, respectively. The results are represented in Table \ref{kclique}. As we can see, there are
    no communities resulting from the application of $5$-Clique.

    \begin{table}[H]
        \centering
        \begin{subtable}{0.60\textwidth}
            \resizebox{\textwidth}{!}{
                \begin{tabular}{| c | c | c | c | c | c | c | c | c | c |}
                    \hline
                    K & Communities & Biggest & Smallest & Modularity & Conductance & IED & AND &  Tags &
                    Langs \\
                    \hline
                    3 & 8 & 12 & 3 & 0.086 & 0.53 & 0.17 & 2.97 & bla & English \\
                    \hline
                    4 & 5 & 4 & 4 & 0.0092 & 0.63 & 0.25 & 3.0 & bla & English \\
                    \hline
                \end{tabular}
            }
        \end{subtable}
        \caption{Evaluation of the partitions obtained by the application of the K-Clique algorithm.}
        \label{kclique}
    \end{table}

% section k_clique (end)

\section{Label Propagation} % (fold)
\label{sec:label_propagation}

% section label_propagation (end)

\section{Louvain} % (fold)
\label{sec:louvain}

% section louvain (end)

\section{Girvan-Newman} % (fold)
\label{sec:girvan_newman}

% section girvan_newman (end)

\section{Demon} % (fold)
\label{sec:demon}

% section demon (end)
