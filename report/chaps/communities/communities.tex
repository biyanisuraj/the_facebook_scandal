In this chapter we'll provide the results obtained by applying \textbf{K-Clique}, \textbf{Label Propagation},
\textbf{Louvain}, \textbf{Girvan-Newman} and \textbf{Demon} to a sample of $1000$ nodes taken from the original
network. We've chosen to sample the crawled data in order to ease the application of the various algorithms.
For each application of the algorithms we provide a table containing the measures for evaluating the obtained
partitions (Modularity, Conductance, IED and AND), the total number of communities discoverd
(Communities), the number of nodes in the smallest/biggest community (Smallest/Biggest), the hashtags utilized
in the biggest community (Tags) and finally the languages of the users in the biggest community (Langs).

\section{K-Clique} % (fold)
\label{sec:k_clique}
    We have chosen to apply the \textbf{K-Clique} algorithm to the sample using three different values for
    $k$: $3$, $4$ and $5$, respectively. The results are represented in Table \ref{kclique}. As we can see, there are
    no communities resulting from the application of $5$-Clique.

    \begin{table}[H]
        \centering
        \begin{subtable}{\textwidth}
            \resizebox{\textwidth}{!}{
                \begin{tabular}{| c | c | c | c | c | c | c | c | c | c |}
                    \hline
                    K & Communities & Biggest & Smallest & Modularity & Conductance & IED & AND &  Tags &
                    Langs \\
                    \hline
                    3 & 8 & 12 & 3 & 0.086 & 0.53 & 0.17 & 2.97 & zuckerberg, cambridgeanalytica,
                    deletefacebook, facebook, facebookgate & English \\
                    \hline
                    4 & 5 & 4 & 4 & 0.0092 & 0.63 & 0.25 & 3.0 & cambridgeanalytica, deletefacebook, facebook
                    & English \\
                    \hline
                \end{tabular}
            }
        \end{subtable}
        \caption{Evaluation of the partitions obtained by the application of the K-Clique algorithm.}
        \label{kclique}
    \end{table}

% section k_clique (end)

\section{Label Propagation} % (fold)
\label{sec:label_propagation}
    In Table \ref{labelprop} are represented the results of the application of the \textbf{Label Propagation}
    algorithm. Together with the Louvain algorithm, it returns the best partition, as we can see by the modularity
    and conductance scores.

    \begin{table}[H]
        \centering
        \begin{subtable}{\textwidth}
            \resizebox{\textwidth}{!}{
                \begin{tabular}{| c | c | c | c | c | c | c | c | c |}
                    \hline
                    Communities & Biggest & Smallest & Modularity & Conductance & IED & AND &  Tags & Langs \\
                    \hline
                    757 & 46 & 1 & 0.66 & 0.24 & 0.18 & 1.42 & zuckerberg, cambridgeanalytica, deletefacebook,
                    facebook ,facebookgate & English \\
                    \hline
                \end{tabular}
            }
        \end{subtable}
        \caption{Evaluation of the partition obtained by the application of the Label Propagaion algorithm.}
        \label{labelprop}
    \end{table}

% section label_propagation (end)

\section{Louvain} % (fold)
\label{sec:louvain}
    The application of the \textbf{Louvain} algorithm returns the best partition among all the partitions returned
    by the other algorithms. The results of its application are represented in Table \ref{louvain}.

    \begin{table}[H]
        \centering
        \begin{subtable}{\textwidth}
            \resizebox{\textwidth}{!}{
                \begin{tabular}{| c | c | c | c | c | c | c | c | c |}
                    \hline
                    Communities & Biggest & Smallest & Modularity & Conductance & IED & AND &  Tags & Langs \\
                    \hline
                    731 & 34 & 1 & 0.75 & 0.070 & 0.14 & 1.62 & cambridgeanalytica, deletefacebook, privacy,
                    zuckerberg, facebook, facebookgate & English, French, Deutsch, Arabic \\
                    \hline
                \end{tabular}
            }
        \end{subtable}
        \caption{Evaluation of the partition obtained by the application of the Louvain algorithm.}
        \label{louvain}
    \end{table}

% section louvain (end)

\section{Girvan-Newman} % (fold)
\label{sec:girvan_newman}

% section girvan_newman (end)

\section{Demon} % (fold)
\label{sec:demon}

% section demon (end)
