In this chapter we'll provide the results obtained by applying \textbf{K-Clique}, \textbf{Label Propagation},
\textbf{Louvain}, \textbf{Girvan-Newman} and \textbf{Demon} to a sample of $1000$ nodes taken from the original
network. We've chosen to sample the crawled data in order to ease the application of the various algorithms.
For each application of the algorithms we provide a table containing the measures for evaluating the obtained
partitions (Modularity, Conductance, IED and AND), the total number of communities discoverd
(Communities), the number of nodes in the smallest/biggest community (Smallest/Biggest), the hashtags utilized
in the biggest community (Tags) and finally the languages of the users in the biggest community (Langs).

\section{K-Clique} % (fold)
\label{sec:k_clique}
    We have chosen to apply the \textbf{K-Clique} algorithm to the sample using three different values for
    $k$: $3$, $4$ and $5$, respectively. The results are represented in Table \ref{kclique}.

    \begin{table}[H]
        \centering
        \begin{subtable}{\textwidth}
            \resizebox{\textwidth}{!}{
                \begin{tabular}{| c | c | c | c | c | c | c | c | c | c |}
                    \hline
                    K & Communities & Biggest & Smallest & Modularity & Conductance & IED & AND &  Tags &
                    Langs \\
                    \hline
                    3 & 21 & 30 & 3 & 0.12 & 0.56 & 0.21 & 2.51 & cambridgeanalytica,
                    deletefacebook, facebook, facebookgate & English \\
                    \hline
                    4 & 1 & 16 & 16 & 0.013 & 0.35 & 0.11 & 6.38 & cambridgeanalytica,
                    deletefacebook, facebook, facebookgate
                    & English \\
                    \hline
                    5 & 2 & 7 & 6 & 0.00015 & 0.67 & 0.23 & 5.05 & cambridgeanalytica, deletefacebook, facebook
                    & English \\
                    \hline
                \end{tabular}
            }
        \end{subtable}
        \caption{Evaluation of the partitions obtained by the application of the K-Clique algorithm.}
        \label{kclique}
    \end{table}

% section k_clique (end)

\section{Label Propagation} % (fold)
\label{sec:label_propagation}
    In Table \ref{labelprop} are represented the results of the application of the \textbf{Label Propagation}
    algorithm.

    \begin{table}[H]
        \centering
        \begin{subtable}{\textwidth}
            \resizebox{\textwidth}{!}{
                \begin{tabular}{| c | c | c | c | c | c | c | c | c |}
                    \hline
                    Communities & Biggest & Smallest & Modularity & Conductance & IED & AND &  Tags & Langs \\
                    \hline
                    714 & 53 & 1 & 0.65 & 0.26 & 0.18 & 1.48 & zuckerberg,
                    cambridgeanalytica, deletefacebook, facebook, facebookgate & English \\
                    \hline
                \end{tabular}
            }
        \end{subtable}
        \caption{Evaluation of the partition obtained by the application of the Label Propagaion algorithm.}
        \label{labelprop}
    \end{table}

% section label_propagation (end)

\section{Louvain} % (fold)
\label{sec:louvain}
    The application of the \textbf{Louvain}, along with the last iteration of the Girvan-Newman algorithm,
    returns the best partition among all the partitions returned by the other algorithms. The results of its
    application are represented in Table \ref{louvain}.

    \begin{table}[H]
        \centering
        \begin{subtable}{\textwidth}
            \resizebox{\textwidth}{!}{
                \begin{tabular}{| c | c | c | c | c | c | c | c | c |}
                    \hline
                    Communities & Biggest & Smallest & Modularity & Conductance & IED & AND &  Tags & Langs \\
                    \hline
                    683 & 51 & 1 & 0.75 & 0.087 & 0.14 & 1.76 & cambridgeanalytica,
                    deletefacebook, facebook, facebookgate & English, Spanish \\
                    \hline
                \end{tabular}
            }
        \end{subtable}
        \caption{Evaluation of the partition obtained by the application of the Louvain algorithm.}
        \label{louvain}
    \end{table}

% section louvain (end)

\section{Girvan-Newman} % (fold)
\label{sec:girvan_newman}
    For the \textbf{Girvan-Newman} algorithm, we've decided to record the results of $5$ iterations over the sample
    network. The results are represented in Table \ref{girvan_newman}. As you can see, the first iteration
    returns a very poor partition, with a low modularity score, due to the fact that the edge with the highest
    betweenness centrality in the starting sample network doesn't provide a good grade of separation among the nodes
    of the network. With the second iteration, and the ones after that, there is a consistent improvement either in
    the modularity score and in the other measures.
    \begin{table}[H]
        \centering
        \begin{subtable}{\textwidth}
            \resizebox{\textwidth}{!}{
                \begin{tabular}{| c | c | c | c | c | c | c | c | c | c |}
                    \hline
                    Iteration & Communities & Biggest & Smallest & Modularity & Conductance & IED & AND &  Tags &
                    Langs \\
                    \hline
                    1 & 673 & 275 & 1 & 0.21 & 0.0049 & 0.20 & 1.40 & cambridgeanalytica,
                    deletefacebook, privacy, zuckerberg, facebook, facebookgate & English, French, Deutsch, Arabic,
                    Spanish, Italian, Portoguese, Hindi, Finnish, Hungarian \\
                    \hline
                    2 & 674 & 165 & 1 & 0.55 & 0.011 & 0.19 & 1.49 & cambridgeanalytica,
                    deletefacebook, privacy, zuckerberg, facebook, facebookgate
                    & English, French, Deutsch, Hindi, Finnish, Spanish \\
                    \hline
                    3 & 675 & 110 & 1 & 0.63 & 0.017 & 0.18 & 1.53 & cambridgeanalytica, deletefacebook, privacy,
                    zuckerberg, facebook, facebookgate & English, French, Deutsch, Arabic, Italian, Portoguese,
                    Hungarian \\
                    \hline
                    4 & 676 & 110 & 1 & 0.66 & 0.025 & 0.17 & 1.57 & cambridgeanalytica, deletefacebook, privacy,
                    zuckerberg, facebook, facebookgate & English, French, Deutsch, Arabic, Italian, Portoguese,
                    Hungarian \\
                    \hline
                    5 & 677 & 87 & 1 & 0.71 & 0.030 & 0.17 & 1.65 & zuckerberg, cambridgeanalytica, deletefacebook,
                    facebook, facebookgate & English, French, Deutsch, Italian, Hindi \\
                    \hline
                \end{tabular}
            }
        \end{subtable}
        \caption{Evaluation of the partition obtained by the application of the Girvan-Newman algorithm.}
        \label{girvan_newman}
    \end{table}

% section girvan_newman (end)

\section{Demon} % (fold)
\label{sec:demon}
    Finally in Table \ref{demon} we provide the results of the application of the \textbf{Demon} algorithm, that
    we tested for three different values of $\epsilon$, $0.25$, $0.50$ and $0.75$, respectively. An interesting
    fact for the application of this algorithm is the one that all the biggest communities discoverd in the three
    applications are composed only by english users. In general, the three partitions are not so good from the
    point of view of the modularity score, with the second application of the algorithm beign the best among the
    three.

    \begin{table}[H]
        \centering
        \begin{subtable}{\textwidth}
            \resizebox{\textwidth}{!}{
                \begin{tabular}{| c | c | c | c | c | c | c | c | c | c |}
                \hline
                Epsilon & Communities & Biggest & Smallest & Modularity & Conductance & IED & AND &  Tags & Langs \\
                \hline
                0.25 & 6 & 30 & 4 & 0.098 & 0.43 & 0.11 & 3.65 & cambridgeanalytica, deletefacebook,
                facebook, facebookgate & English \\
                \hline
                0.50 & 9 & 30 & 4 & 0.085 & 0.49 & 0.12 & 4.05 & cambridgeanalytica, deletefacebook, facebook,
                facebookgate & English \\
                \hline
                0.75 & 13 & 27 & 4 & 0.082 & 0.51 & 0.14 & 3.86 & cambridgeanalytica, deletefacebook,
                facebook, facebookgate & English \\
                \hline
                \end{tabular}
            }
        \end{subtable}
        \caption{Evaluation of the partition obtained by the application of the Demon algorithm.}
        \label{demon}
    \end{table}

% section demon (end)
