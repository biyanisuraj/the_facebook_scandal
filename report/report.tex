\documentclass[11pt, twoside]{report}

\usepackage[T1]{fontenc}
\usepackage[utf8]{inputenc}
%%\usepackage[us]{babel}
\usepackage{helvet}
\usepackage[a4paper,width=180mm,top=20mm,bottom=20mm,bindingoffset=6mm]{geometry}
\usepackage[framemethod=default]{mdframed}
\usepackage{lipsum,titlesec,xcolor,fancyhdr,array,multicol,float,graphicx,wrapfig,subcaption}
\usepackage{fontspec,todonotes,enumitem,comment,capt-of,amsmath,booktabs}
\usepackage[colorlinks]{hyperref}
\usepackage[font=scriptsize]{caption}
\usepackage[sorting=none]{biblatex}
\addbibresource{references.bib}


\DeclareCaptionLabelFormat{andtable}{#1˜#2 \& \tablename˜\thetable}

\renewcommand{\familydefault}{\sfdefault}

\graphicspath{{images/}}

%% define size for chapter initial page:
\usepackage{titlesec, blindtext, color}
\definecolor{gray75}{gray}{0.75}
\newcommand{\hsp}{\hspace{20pt}}
\titleformat{\chapter}[hang]{\Huge\bfseries}{\thechapter\hsp\textcolor{gray75}{|}\hsp}{0pt}{\Huge\bfseries}

\titlespacing*{\chapter}{0pt}{-60pt}{20pt} %% change spacing

\usepackage{enumitem}
\setlist{noitemsep}
%%%%%%%%%%


\floatstyle{plain}
\restylefloat{figure}

\pagestyle{fancy}
\fancyhf{}
\renewcommand{\chaptermark}[1]{ \markboth{#1}{} }
\renewcommand{\sectionmark}[1]{ \markright{#1}{} }
\fancyhead[LE]{\itshape{ \fontsize{10}{12} \selectfont \leftmark}}
\fancyhead[RE, LO]{\thepage}
\fancyhead[RO]{\itshape{\fontsize{10}{12} \selectfont \rightmark}}
\renewcommand\headrulewidth{0pt}

 \hypersetup{
     colorlinks,
     citecolor= green,
     filecolor= black,
     linkcolor= black,
     urlcolor= blue
 }

\usepackage[nottoc,notlot,notlof]{tocbibind}
 
\begin{document}
    \begin{titlepage}
    \centering
    \includegraphics[width=0.30\textwidth]{images/cherubino_black.pdf}\par\vspace{1cm}
    {\scshape\LARGE Università di Pisa \par}
    \vspace{1cm}
    {\scshape\ Social Network Analysis \\A.A. 2017/2018\par}
    \vspace{1.5cm}
    {\huge\bfseries Cambridge Analytica and Facebook: \\ The Scandal and the Fallout on Twitter\\ \par}
    \vspace{2cm}
    {\Large Gianmarco Ricciarelli 555396 \\ Stefano Carpita 304902 \par}
    \vfill

\begin{flushright}    
  \textit{Data drives all we do.}\\
\vspace{2 mm}
Cambridge Analytica main slogan.
\end{flushright}

\vspace{4 mm}
    
\begin{flushright}    
  \textit{Rules don’t matter for them. \\ For them, this is a war, and it’s all fair.}\\
\vspace{2 mm}
  Christopher Wylie, \\ former datascientist at Cambridge Analytica, about its leaders.
\end{flushright}




  \end{titlepage}


    \pagenumbering{Roman}
    \tableofcontents
    \chapter{The case story}
    \pagenumbering{arabic}

On Saturday $17^{th}$ of March 2018, The New York Times and The Guardian / The Observer broke reports on how the consulting firm Cambridge Analytica harvested private information from the Facebook profiles of more than 50 million users without their permission, making it one of the largest data leaks in the social network’s history \cite{guardian}, \cite{nyt_17march}.

Cambridge Analytica described itself as a company providing consumer research, targeted advertising and other data-related services to both political and corporate clients. The whistleblower Christopher Wylie, datascientist and former director of research at Cambridge Analytica revealed to the Observer how Cambridge Analytica used personal information taken without authorisation in early 2014 to build a system that could profile individual US voters, in order to target them with personalised political advertisements.
Christopher Wylie, who worked with a Cambridge University academic to obtain the data, told the Observer: “We exploited Facebook to harvest millions of people’s profiles. And built models to exploit what we knew about them and target their inner demons. That was the basis the entire company was built on.”


We report here a timeline of the events of the first days of the scandal, from different sources \cite{nyt_timeline}:
\begin{itemize}

\item March 17: The Observer and The New York Times publish joint reports on data harvesting by Cambridge Analytica. UK Information Commissioner Elizabeth Denham issues statement that they are “investigating circumstances in which Facebook data may have been illegally acquired and used.” Politicians in US and UK call for investigation.

\item March 19: Channel 4 News publishes part 1 of their undercover investigation into Cambridge Analytica. Facebook sends investigators to Cambridge Analytica’s offices. UK Information Commissioner orders them to stand down. \cite{channel4}

\item March 20: Channel 4 News publishes part 2 of their undercover investigation into Cambridge Analytica, where they boast about getting Donald Trump elected. British MP Damian Collins calls on Facebook to present oral evidence on Cambridge Analytica. Facebook agrees to send former operations manager Sandy Parakilas. Facebook holds internal Q\&A with attorney Paul Grewal to discuss the crisis, but CEO Mark Zuckerberg and COO Sheryl Sandberg do not attend. Cambridge Analytica suspends CEO Alexander Nix. Facebook demands to inspect Christopher Wylie’s phone. FTC opens investigation into Facebook.
\item March 21,
  Brian Acton, co-founder of the messaging app WhatsApp, called on his Twitter followers to \#deletefacebook. Facebook purchased WhatsApp in 2014.
  \begin{center}
  @brianacton: It is time. \#deletefacebook\\    
  \end{center}
In a Facebook post published several days after the initial reports, Zuckerberg responded to the continued fallout over the data scandal.
He said:\\
\textit{"We have a responsibility to protect your data, and if we can't then we don't deserve to serve you. I've been working to understand exactly what happened and how to make sure this doesn't happen again."}  
\item March 23,
  Elon Musk joined the \#DeleteFacebook movement, taking down official pages for two of his companies, Tesla and SpaceX, announcing his decision on Twitter. Both the SpaceX and Tesla accounts vanished within hours of his tweet.
\item March 24, 
  The Ars Technica website reported that Facebook ``surreptitiously'' collected call and SMS data for years from Android users, including names, phone numbers, and the length of calls \cite{ars_technica}.
\end{itemize}








    \chapter{Building the network}
    In this report we present an analysis of the spreading of the Cambridge Analytica-Facebook scandal on Twitter.
    We have considered a network composed by the authors of tweets about the case, during the first period of the scandal outbreak. The data have been collected via the Twitter API and we built the network using the following consecutive steps:
    \begin{enumerate}
    \item Crawling of all the available tweets over a period of more than 15 days, since the $17^{th}$ of March, containing at least one of the most popular hashtags regading the case:\\
    \textbf{  \{\#cambridgeanalytica, \#facebookgate, \#deletefacebook, \#zuckerberg \} }
      
    \item Cleaning of the crawled tweets, by selecting and storing in a MongoDB database only the users informations about the authors of tweets, excluding retweets and mentions.
    \item Selection of the case outbreak time period by observing the time history in Fig. \ref{fig:time_history}. The selected time period consists of 8 days, from the $17^{th}$ to the $24^{th}$ of March included (considering the Italian timezone). 
      
    \item Crawling of the following list for each of the selected authors, extracting the followee/follower relationships.
    \end{enumerate}
    In Fig. \ref{fig:time_history} is represented the rate of new authors per hour. %%In the weekend of $17^{th}$ $18^{th}$
    We can observe a tipical daily periodicity, with peaks during the Italian afternoon, corresponding to the USA waking up.
    The first relevant rate increase is observed on the Monday $19^{th}$, following the weekend of news publishing. A stationary evolution is observed after the $24^{th}$ of March, a week after the scandal outburst.
    %%Relevant peaks are in correspondance on major events likewise...

    The collecting of the following lists of all the authors has been necessary in order to overcome the huge time complexity of a direct crawling of the
    followee/follower relationships, considering also the Twitter API rates limits. A negligible fraction of the following lists was not available, because of the Twitter privacy settings of the users.

    In summary, we built a directed unweighted network, consisting of:

    \begin{itemize}
    \item \textbf{Nodes}: $65729$ twitter users, authors of tweets containing at least one of the hashtags specified above.

    \item \textbf{Edges}: $2501757$ followee/follower relationships between the selected authors.
    \end{itemize}

    
    \begin{figure}[htbp]
      \centering
\includegraphics[width=\textwidth]{../../scripts/network_analysis/imgs/time_history.pdf}
      \caption{Time history of the rate of new authors per hour, during the first 12 days from the first publishing of the scandal.}
      \label{fig:time_history}
    \end{figure}



    
    \chapter{Network analysis}
\section{Network properties summary} 


The properties of the networks analyzed are summarized in Tab. \ref{tab:summary}. The $g$ network represents the original directed graph, while
$g_{und}$ is the graph obtained by not considering the directivity of the links. We observe that the number of links of $g_{und}$ is lower respect to the one of $g$: the difference of $605879$ represents the number of reciprocal links, corresponding to the number of user pairs following each other, a $24\%$ fraction of the overall number of authors.

The network has been compared throughout the report with two synthetic network, an Erdos-Renyi random network $g_{ER}$ and a Barabasi-Albert network $g_{BA}$. The Erdos-Renyi random network, with directed connections, has been generated with a value of the ``linking probability'' $p$ computed using the double of the average degree of the original network:
    \begin{equation}
      p_{ER} \approx \frac{2\langle k  \rangle}{N} = \frac{76}{65729} \approx  0.001
      \label{eq:ER_probability}
    \end{equation}

 An undirected network built with the Barabasi-Albert model  has an average degree equal to the double of the links formed by each new node: $ \, \langle k_{BA} \rangle = 2 m$ \cite{network_science}. Because our network is directed we generated a Barabasi-Albert random network with a value of $m$ equal to half of the double of the average degree of the original network, or simply equal to the average degree of our directed graph:
    \begin{equation}
      m = \frac{ 2 \, \langle k \rangle}{2} =  \langle k \rangle = 38
      \label{eq:BA_model}
    \end{equation}
    The network $g_{Ita}$ it is the subgraph of $g$ including only the Italian users, who have been identified by using the language metadata available with the crawled tweets.


\begin{table}[htbp]
  \centering
  \begin{tabular}{lllll}
\toprule
{} &        g &    g\_und &     g\_er &     g\_ba \\
\midrule
L         &  2501757 &  1895878 &  4318406 &  4989628 \\
N         &    65729 &    65729 &    65729 &    65729 \\
density   &  0.00058 &  0.00088 &    0.001 &  0.00231 \\
gamma     &      2.6 &     None &     None &      2.9 \\
gamma\_in  &      2.4 &     None &     None &     None \\
gamma\_out &      2.9 &     None &     None &     None \\
gamma\_tot &      2.6 &     None &     None &     None \\
k\_avg     &       38 &       57 &       65 &      151 \\
k\_in\_max  &    19064 &     None &      109 &     None \\
k\_in\_min  &        0 &     None &       36 &     None \\
k\_max     &    19073 &    19065 &      183 &     3640 \\
k\_min     &        1 &        1 &       84 &       76 \\
k\_out\_max &     4130 &     None &      103 &     None \\
k\_out\_min &        0 &     None &       35 &     None \\
\bottomrule
\end{tabular}

  \label{tab:summary}
  \caption{Label}
\end{table}

The network is composed by a large strongly connected component including about the $75\%$ of the original graph, as depicted in Fig. \ref{fig:components}. The remaining nodes represents users who follows only few authors of the graph. Without considering the directivity of the links
we have a weakly component including almost the whole network. The generated random graphs are completely connected.


    \begin{figure}[H]
      \centering
      \includegraphics[width=\textwidth]{../../scripts/network_analysis/imgs/connectivity.pdf}            
      \caption{Weakly and strongly connected components.}
      \label{fig:components}
    \end{figure}


    \section{Degree distributions}
    The analysis of the degree distribution of a network permits to identify the characteristic power law trend, immediately recognizable
    by using a log-log scale plot. Observing Fig. \ref{fig:degree_nolog} it is evident the difference between the degree distributions of the original graph respect to the random generated Erdos-Renyi graph.
    The log scale used requires to use a logarithimc binning of the degrees values: in Figs. \ref{fig:degree_comparison}, \ref{fig:tot_degree}, \ref{fig:in_degree}, \ref{fig:out_degree}, we used a bin size increasing as $2^k$.
    The distribution of the total, in and out degrees shows the presence of two different regimes, the free-scale property becomes evident
    for a degree larger than few hundreds. In order to compute the value of the power law exponent $\gamma$ of Eq. \ref{eq:power_law} we used an ordinary least squares regression, visually identifying the regime zones, and checking for minimal variations of the obtained $\gamma$, by progressively widening the degree interval chosen for the regression.

    \begin{equation}
      p(k) = C k^{-\gamma}
      \label{eq:power_law}
    \end{equation}

    The out-degree distribution shows a trend very close to the Barabasi-Albert model, with an exponent $\gamma_{out} = 2.9$, the same value obtained for the generated BA graph. The in-degree distribution decreases with slower speed, having $\gamma_{in} = 2.4$. The total degree distribution exponent has a value of $\gamma_{tot} = 2.6$, intermediate between the in and out distributions.
 

 \begin{figure}[hb]
      \centering
      \includegraphics[width=0.95\textwidth]{../../scripts/network_analysis/imgs/degree_distributions_nobinlog.pdf}            
      \caption{Comparison of the degree distributions, plotted without using logarithmic binning.}
      \label{fig:degree_nolog}
    \end{figure}


\begin{minipage}[b]{0.5\textwidth}
   \centering
    \begin{figure}[H]
      \includegraphics[width=\textwidth]{../../scripts/network_analysis/imgs/degree_distributions.pdf}            
          \caption{Comparison of degree distributions.}
        \label{fig:degree_comparison}
\end{figure}
\end{minipage}
\begin{minipage}[b]{0.5\textwidth}
  \begin{figure}[H]
  \centering
  \includegraphics[width=\textwidth]{../../scripts/network_analysis/imgs/tot_degree_distribution.pdf}            
        \caption{Total degree distribution.}
\label{fig:tot_degree}
\end{figure}
\end{minipage}
    
\begin{minipage}[b]{0.5\textwidth}
   \centering
    \begin{figure}[H]
      \includegraphics[width=\textwidth]{../../scripts/network_analysis/imgs/in_degree_distribution.pdf}            
          \caption{In-degree distribution.}
        \label{fig:in_degree}
\end{figure}
\end{minipage}
\begin{minipage}[b]{0.5\textwidth}
  \begin{figure}[H]
  \centering
  \includegraphics[width=\textwidth]{../../scripts/network_analysis/imgs/out_degree_distribution.pdf}            
        \caption{Out-degree distribution.}
\label{fig:out_degree}
\end{figure}
\end{minipage}




    \section{Path analysis}
    In order to exactly estimate the average path length $\langle d \rangle$ it would be necessary to compute all the node-node distances of the network. These procedure results infeasible with the computation resources available, as shown in Fig. \ref{fig:path_time}.
    In real networks the path length distribution is quite close to a normal distribution, as shown in \cite{ye_paths}. The average path length has then been estimated statistically, random sampling a number $n$ of node pairs, sufficient to achieve a narrow confidence interval for the mean. The assumption of normality of the distribution it is strong, but not necessary. The convergence of the computed mean to the expected value is guaranteed by the central limit theorem with the assumptions that the distances are independent, identically distributed, and with finite variance.
    The average path length has been estimated by the average of the distances $D_i$ for each sampled node pair, and computing its standard deviation:
    \begin{equation}
      \langle d \rangle = \frac{\sum D_i}{n} \; , \; \sigma(\langle d \rangle) = \frac{s}{\sqrt{n}}
    \end{equation}

    
\begin{minipage}[b]{0.5\textwidth}
   \centering
    \begin{figure}[H]
      \includegraphics[width=\textwidth]{../../scripts/network_analysis/imgs/paths_computation_time.pdf}            
          \caption{Shortest paths computation time by number of pairs}
      \label{fig:path_time}
\end{figure}
\end{minipage}
\begin{minipage}[b]{0.5\textwidth}
  \begin{figure}[H]
  \centering
      \includegraphics[width=\textwidth]{../../scripts/network_analysis/imgs/paths_hist.pdf}            
        \caption{Shortest paths distribution}
\end{figure}
\end{minipage}

The average shortest path length, computed on the undirected graph, it is equal to 2.92, close to the average distance
obtained for both the random graphs, as shown in Fig. \ref{fig:path_comparison}.
The shortest path distribution for the original network has a larger dispersion respect to the random ones, there are shortest paths
reaching a length equal to 8.


    
    \begin{figure}[htbp]
      \centering
      \includegraphics[width=\textwidth]{../../scripts/network_analysis/imgs/paths_hist_comparison.pdf}            
      \caption{Shortest paths distributions comparison between the original undirected graph and the random networks.}
      \label{fig:path_comparison}
    \end{figure}



\begin{minipage}[b]{0.5\textwidth}
   \centering
    \begin{figure}[H]
      \includegraphics[width=\textwidth]{../../scripts/network_analysis/imgs/cluster_coef_hist.pdf}            
          \caption{Clustering coefficient distribution}
      \label{fig:path_time}
\end{figure}
\end{minipage}
\begin{minipage}[b]{0.5\textwidth}
  \begin{figure}[H]
  \centering
      \includegraphics[width=\textwidth]{../../scripts/network_analysis/imgs/cluster_coef_bydegree.pdf}            
      \caption{Clustering coefficients as function of the degree}
\end{figure}
\end{minipage}

\clearpage
\section{Hubs analysis}

The biggest hubs of the crawled social network of tweets authors about the Cambridge Analytica-Facebook scandal are mainly
news mass media, as expected. In Fig. \ref{fig:hubs_followers} are represented the 30 biggest hubs (by in-degree) by indicating
the in-degree of the crawled network, corresponding to the number of authors following the hub, versus the actual total
number of followers on Twitter.
The "The New York Times" is the biggest hub, with the maximum number of both in-degree and number of followers.
We observe that there is a positive correlation between in-degree and followers, with some variations.
In particular, let's take a pair of hubs having similar followers count, such as the "Washington Post" and the "Huffington Post".
The "Washington Post" has a larger in-degree than the second.
This difference can be interpreted as a larger interest in the scandal from the people following the "Washington Post'' respect 
to the ones following the "Huffington Post".
We can define a quantity to measure this interest:

\begin{equation}
  \text{Interest} \equiv \frac{ \text{in-degree} }{\text{\#followers}}
  \label{eq:interest}
\end{equation}

This measure represents the fraction of followers that being interested in the scandal had published a tweet about the subject.
%%It is also a measure of density, being the analyzed network a sub-network of the overall Twitter network: 
We can observe for example a large difference in Interest between the two earliest sources, The Guardian and The New York Times,
meaning that the followers of the ``The Guardian'' have relatively interacted much more about the case.

Furthermore we analyzed the correlation of the Interest with the measures in Fig. \ref{fig:interest_corr_hist}, for about 300 of the biggest hubs.
The Interest shows an high positive correlation with the Clustering Coefficient, with a value $r=0.78$, and a mildy negative correlation with the in-degree and the number of followers. The correlation is almost equal to zero respect to the closeness centrality $cc$.

The correlation between Interest and the Clustering Coefficient $C$ can be interpreted
by considering that nodes with higher $C$ have denser connections, that may strengthen the Interest with an high reinforce from the
direct neighbours.


    \begin{figure}[htbp]
      \centering
      \includegraphics[width=\textwidth]{../../scripts/network_analysis/imgs/hubs_followers.pdf}            
      \caption{}
      \label{fig:hubs_followers}
    \end{figure}

\begin{minipage}[b]{0.5\textwidth}
   \centering
    \begin{figure}[H]
      \includegraphics[width=\textwidth]{../../scripts/network_analysis/imgs/hubs_interest_corr.pdf}            
          \caption{Interest versus clustering coefficient for about 300 of the biggest hubs.}
      \label{fig:interest_corr}
\end{figure}
\end{minipage}
\begin{minipage}[b]{0.5\textwidth}
  \begin{figure}[H]
  \centering
  \includegraphics[width=\textwidth]{../../scripts/network_analysis/imgs/hubs_interest_corr_hist.pdf}
  \caption{Correlations of the Interest with other measures.}
  \label{fig:interest_corr_hist}
\end{figure}
\end{minipage}

    
    \begin{figure}[htbp]
      \centering
      \includegraphics[width=\textwidth]{../../scripts/network_analysis/imgs/hubs_interest.pdf}            
      \caption{}
%      \label{fig:path_comparison}
    \end{figure}

\newpage    
\section{Italian sub network}
Some of the analysis on the original graph have been repeated on the Italian sub-network, and are here presented.
\begin{minipage}[b]{0.5\textwidth}
   \centering
    \begin{figure}[H]
      \includegraphics[width=\textwidth]{../../scripts/network_analysis/imgs/tot_degree_distribution_ita.pdf}            
          \caption{Total degree distribution.}
      \label{fig:tot_degree_ita}
\end{figure}
\end{minipage}
\begin{minipage}[b]{0.5\textwidth}
  \begin{figure}[H]
  \centering
      \includegraphics[width=\textwidth]{../../scripts/network_analysis/imgs/paths_hist_ita.pdf}            
        \caption{Shortest paths distribution.}
\end{figure}
\end{minipage}

    \begin{figure}[htbp]
      \centering
      \includegraphics[width=\textwidth]{../../scripts/network_analysis/imgs/hubs_followers_ita.pdf}            
      \caption{}
      \label{fig:hubs_followers_ita}
    \end{figure}

    \begin{figure}[htbp]
      \centering
      \includegraphics[width=\textwidth]{../../scripts/network_analysis/imgs/hubs_interest_ita.pdf}            
      \caption{}
      \label{fig:hubs_followers_ita}
    \end{figure}

    


    \chapter{Spreading} % (fold)
\label{cha:spreading}

\section{Which model for news spreading?}

In order to choose a model for the spreading of news or ideas it's necessary to point out
what means in this field an infection, and to interpret the related quantities.
A person "infected'' by a news it's not a person just only reached by the news, but it's a person
who partecipates to the spreading by communicating to its neighbours and potentially infect them.
The decision to spread the information it's an individual choice, expressing the interest
of the person to the news, and indicates the presence of a personal threshold of reaction related to the news or ideas.
The threshold can also be influenced by the neighbourhood or the community membership.


In the epidemic models the coefficient $\beta$ represents the rate of trasmission of the infection and it is constant for the whole population, indicating the dependence on the properties of the pathogen.
In the news field $\beta$ can be interpreted as an intrinsic power of trasmission of the news.
This power of trasmission may be associated to the journalistic concept of \textit{newsworthiness},
which includes all the characteristics that make a fact a worthy news.
But the expanding phenomenon of \textit{fake news} shows that the speed of diffusion it's not only related to
the worthiness of an information. A recent paper published on Science \cite{Vosoughi_2018} shows how false news on Twitter spread 
``significantly farther, faster, deeper, and more broadly than the truth in all categories of information''.
The authors of the paper tried to explain the faster speed of diffusion of the false news by its novelty and the conveying
of strongest emotive reactions like surprise or disgust.
A news coverage is usually characterized also by a certain amount of time after which the news naturally ``dies'' out.
The SI and SIS epidemic models applied to free-scale networks predict asymptotically scenarios where there is always a finite 
fraction of infected nodes. From this point of view the SIR model may fit better to reality, predicting the vanishing of the news trasmission.
The coefficient $\mu$ of the epidemic models, likewise $\beta$, it's constant and in this case may represent the intrinsic property of a news to vanish, making the people stopping communicating about it. In summary:

\begin{itemize}
\item pathogen or agent: the news or information being spread.
\item infected node: a node that is communicating a news, for a certain period of time.
\item trasmission rate $\beta$: intrinsic power of trasmission of a news/information.
\item recover rate $\mu$: intrinsic rate related to a news at which the infected nodes stop communicating about it.
\item immunization: the node stops to communicate about the news definitely.
\item threshold: personal decision to communicate about the news or not, dependent or not on the neighbours decisions.
\end{itemize}


In this chapter we'll describe the results we obtained by applying the \textbf{SI}, \textbf{SIS}, \textbf{SIR},
and \textbf{Threshold} diffusion models both on the crawled data and on the synthetic graphs (Erdős–Rényi and
Barabási–Albert).

\section{SI model} % (fold)
\label{sec:si_model}


\begin{figure}[htbp]
  \centering
  \begin{subfigure}{0.45\textwidth}
    \resizebox{\textwidth}{!}{
      \includegraphics{images/spreading/si/trend_comparison_01.pdf}
    }
            \caption{Infection rate $\beta_0= 0.01$.}
            \label{fig:diff_si_1}
        \end{subfigure}
        \begin{subfigure}{0.45\textwidth}
            \resizebox{\textwidth}{!}{
              \includegraphics{images/spreading/si/trend_comparison_001.pdf}
            }
            \caption{Infection rate $\beta_1= \frac{\beta_0}{10} =  0.001$. }
            \label{fig:diff_si_2}
          \end{subfigure}
          \caption{Comparison of two trasmission rates in the SI model, for the original network $G$, the Erdos-Renyi $ER$ and the Barabasi-Albert $BA$.}
          \label{fig:diff_si}
     \end{figure}

For the \textbf{Susceptible-Infected} model we've started with a random $0.005\%$ of the total population ($3$ nodes)
    of each network being infected, representing the earliest sources of information.
    In Fig. \ref{fig:diff_si} we compare two different trasmission rates, with an order of magnitude of difference:
    $ \beta_0 =0.01$ for   $\beta_1 = 0.001$. The original network asymptotically reach the saturation regime only for the fraction of nodes
    in the strong connected component.


% section si_model (end)

\section{SIS model} % (fold)
\label{sec:sis_model}
    \begin{figure}[H]
        \centering
        \begin{subfigure}{0.3\textwidth}
            \resizebox{\textwidth}{!}{
                \includegraphics{images/spreading/sis/diffusion_G_comparison.pdf}
            }
            \caption{Original network}
            \label{diff_sis}
        \end{subfigure}
        \begin{subfigure}{0.3\textwidth}
            \resizebox{\textwidth}{!}{
                \includegraphics{images/spreading/sis/diffusion_BA_comparison.pdf}
            }
            \caption{Barabasi-Albert network}
            \label{diff_sis_er}
        \end{subfigure}
        \begin{subfigure}{0.3\textwidth}
            \resizebox{\textwidth}{!}{
                \includegraphics{images/spreading/sis/diffusion_ER_comparison.pdf}
            }
            \caption{Erdos-Renyi network}
            \label{diff_sis_ba}
        \end{subfigure}
        \caption{Comparison of the SIS model for the original network $G$, the Erdos-Renyi $ER$ and the Barabasi-Albert $BA$.}
        \label{fig:diff_sis_comparison}
      \end{figure}

      The introduction of the recovery rate $\mu$ in the \textbf{Susceptible-Infected-Susceptible} model for networks epidemics
      provides an epidemic threshold $\lambda_C$ for the spreading rate $\lambda$, dependent on the second order moment of the degree
      distribution $\langle k^2 \rangle$.
      For a random network the epidemic threshold given by Eq. \ref{eq:epidemic_threshold} is finite, and defines two possible asymptotically outcomes, an \textbf{endemic state} characterized by a finite fraction of infected individuals, and a completely \textbf{disease free} state.
      \begin{equation}
        \lambda_C(ER) = \frac{1}{\langle k \rangle +1 } \Rightarrow \lambda>\lambda_C:\: \text{epidemic state}
        \label{eq:epidemic_threshold}
      \end{equation}
      The free-scale networks are characterized by a diverging variance, which means the epidemic threshold tends to vanish, causing a finite fraction of infected individuals also for small $\lambda$. 
      We used the random network threshold to choose the recovery rate to simulate both the possible states. We observe in Fig. \ref{fig:diff_sis_comparison} how for the free-scale networks the infected fraction is always finite, below and above the epidemic threshold, while for the random network we observe also the disease free state.
      
% section sis_model (end)

\section{SIR model} % (fold){}
\label{sec:sir_model}
    The key characteristic of the \textbf{Susceptible-Infected-Recovered} model consists in the possibility
    of the individuals to recover from the disease and hence to be ``removed'' from the population instead of returning to the susceptible state.
    We tested this model either for the case in which $\mu$ is smaller than $\beta$ and the other way around.
    The different situations are represented in Fig. \ref{fig:diff_sir_total}, we observe the typical vanishing of the fraction of the infected nodes, after a steep initial rise similar to the one described by the SI model.

    \begin{figure}
        \begin{subfigure}{0.45\textwidth}
            \resizebox{\textwidth}{!}{
                \includegraphics{images/spreading/sir/diffusion_smaller.pdf}
            }
            \caption{Original network}
            \label{diff_sir_smaller}
        \end{subfigure}
        \begin{subfigure}{0.45\textwidth}
            \resizebox{\textwidth}{!}{
                \includegraphics{images/spreading/sir/diffusion_greater.pdf}
            }
            \caption{Original network}
            \label{diff_sir_greater}
        \end{subfigure}
        \begin{subfigure}{0.45\textwidth}
            \resizebox{\textwidth}{!}{
                \includegraphics{images/spreading/sir/diffusion_er_smaller.pdf}
            }
            \caption{Erdos-Renyi network.}
            \label{diff_sir_er_smaller}
        \end{subfigure}
        \begin{subfigure}{0.45\textwidth}
            \resizebox{\textwidth}{!}{
                \includegraphics{images/spreading/sir/diffusion_er_greater.pdf}
            }
            \caption{Erdos-Renyi network.}
            \label{diff_sir_er_greater}
        \end{subfigure}
        \begin{subfigure}{0.45\textwidth}
            \resizebox{\textwidth}{!}{
                \includegraphics{images/spreading/sir/diffusion_ba_smaller.pdf}
            }
            \caption{Barabasi-Albert network.}
            \label{diff_sir_ba_smaller}
        \end{subfigure}
        \begin{subfigure}{0.45\textwidth}
            \resizebox{\textwidth}{!}{
                \includegraphics{images/spreading/sir/diffusion_ba_greater.pdf}
              }
            \caption{Barabasi-Albert network.}
            \label{diff_sir_ba_greater}
        \end{subfigure}
        \caption{SIR model applied to the networks, with two different $\lambda$ values.}
        \label{fig:diff_sir_total}
    \end{figure}

% section sir_model (end)

\section{Threshold model} % (fold)
\label{sec:threshold_model}

    In order to test the \textbf{Threshold model} we've chosen a threshold $\tau$ equal to $0.10$. The
    diffusion of the infection for this model is represented in Figure \ref{diff_thr_total}. As we can see, for the
    original network we have that almost all the nodes become infected within the first $20$ model's iterations,
    due to the fact that the value chosen for the threshold results sufficient for the spreading of the
    infection. If we change the threshold's value, this time using a value of $0.20$, we can observe that the
    original network become immune to the infection, thanks to its internal structure. We can observe
    the same immunity in the Erdős–Rényi and Barabási–Albert network for the original threshold's value.



    \begin{figure}[htbp]
        \begin{subfigure}{0.45\textwidth}
          \resizebox{\textwidth}{!}{
            \includegraphics{images/spreading/threshold/trend_comparison_infection.pdf}
            }
            \caption{Endemic state with lower threshold, $\tau=0.1$}
        \end{subfigure}
        \begin{subfigure}{0.45\textwidth}
            \resizebox{\textwidth}{!}{
              \includegraphics{images/spreading/threshold/trend_comparison_disfree.pdf}
            }
            \caption{Disease-free state with higher threshold, $\tau=0.2$ }
          \end{subfigure}
          \caption{Threshold model applied to the networks, the Barabasi trend is superimposed by the Erdos-Renyi one.}
      \label{diff_thr_total}
  \end{figure}


      




% section threshold_model (end)


\section{The New York Times vs La Repubblica vs Sputnik Italia}

Let's assume that the same news it is initially spread by 3 different sources, with very different degree values:
\begin{itemize}
\item The New York Times, the biggest hub,  with in-degree $k_{in}=19064$  and \#followers=41595294.
\item La Repubblica, with $k_{in}=961$ and \#followers=2843075.
\item Sputnik Italia, with $k_{in}=71$ and \#followers=6490.  
\end{itemize}
We applied the SIR model obtaining two scenarios, a viral news reaching a large part of the network and a minor news vanishing 
out after a mildly spread. The scenarios depicted in Fig. \ref{fig:viral_news} have been obtained maintaining the same transmission rate $\beta$ and changing the recovery rate $\mu$.
In the viral news scenario the different degree values caused only a modest translation in time of the spread trends, with the same percentages
of infected nodes. This result may be explained clearly by the properties of the free-scale networks: the presence of the hubs and the ultra small world property cause a fast spreading from @sputnik\_italia to the hubs, who infect immediately a large fraction of the network.
In the second scenario, by decreasing the $\lambda$ ratio below a certain threshold, the news spread by @sputnik\_italia nodes probably did not reach the hubs and the infection did not propagate.

The two presented scenarios can explain the fact that for example \textit{fake news} can also begin spreading from peripheral, small nodes and propagates virally over a network. The difference between the two scenarios is carried by the intrinsic transmission properties of the news,
independently from the properties of the carrier: the message it is more important than the messenger.

   \begin{figure}[H]
        \centering
        \begin{subfigure}{0.45\textwidth}
            \resizebox{\textwidth}{!}{
                \includegraphics{images/spreading/news_viral.pdf}
            }
            \caption{}
            \label{diff_thr}
        \end{subfigure}
        \begin{subfigure}{0.45\textwidth}
            \resizebox{\textwidth}{!}{
                \includegraphics{images/spreading/news_minor.pdf}
            }
            \caption{}
            \label{diff_thr_er}
          \end{subfigure}
          \caption{Viral news spreading: infected percentages trends.}
          \label{fig:viral_news}
\end{figure}


    
% chapter spreading (end)


    In this chapter we'll provide the results obtained by applying \textbf{K-Clique}, \textbf{Label Propagation},
\textbf{Louvain}, \textbf{Girvan-Newman} and \textbf{Demon} to a sample of $1000$ nodes taken from the original
network. We've chosen to sample the crawled data in order to ease the application of the various algorithms.
For each application of the algorithms we provide a table containing the measures for evaluating the obtained
partitions (Modularity, Conductance, IED and AND), the total number of communities discoverd
(Communities), the number of nodes in the smallest/biggest community (Smallest/Biggest), the hashtags utilized
in the biggest community (Tags) and finally the languages of the users in the biggest community (Langs).

\section{K-Clique} % (fold)
\label{sec:k_clique}
    We have chosen to apply the \textbf{K-Clique} algorithm to the sample using three different values for
    $k$: $3$, $4$ and $5$, respectively. The results are represented in Table \ref{kclique}. As we can see, there are
    no communities resulting from the application of $5$-Clique.

    \begin{table}[H]
        \centering
        \begin{subtable}{\textwidth}
            \resizebox{\textwidth}{!}{
                \begin{tabular}{| c | c | c | c | c | c | c | c | c | c |}
                    \hline
                    K & Communities & Biggest & Smallest & Modularity & Conductance & IED & AND &  Tags &
                    Langs \\
                    \hline
                    3 & 8 & 12 & 3 & 0.086 & 0.53 & 0.17 & 2.97 & zuckerberg, cambridgeanalytica,
                    deletefacebook, facebook, facebookgate & English \\
                    \hline
                    4 & 5 & 4 & 4 & 0.0092 & 0.63 & 0.25 & 3.0 & cambridgeanalytica, deletefacebook, facebook
                    & English \\
                    \hline
                \end{tabular}
            }
        \end{subtable}
        \caption{Evaluation of the partitions obtained by the application of the K-Clique algorithm.}
        \label{kclique}
    \end{table}

% section k_clique (end)

\section{Label Propagation} % (fold)
\label{sec:label_propagation}
    In Table \ref{labelprop} are represented the results of the application of the \textbf{Label Propagation}
    algorithm. Together with the Louvain algorithm, it returns the best partition, as we can see by the modularity
    and conductance scores.

    \begin{table}[H]
        \centering
        \begin{subtable}{\textwidth}
            \resizebox{\textwidth}{!}{
                \begin{tabular}{| c | c | c | c | c | c | c | c | c |}
                    \hline
                    Communities & Biggest & Smallest & Modularity & Conductance & IED & AND &  Tags & Langs \\
                    \hline
                    757 & 46 & 1 & 0.66 & 0.24 & 0.18 & 1.42 & zuckerberg, cambridgeanalytica, deletefacebook,
                    facebook ,facebookgate & English \\
                    \hline
                \end{tabular}
            }
        \end{subtable}
        \caption{Evaluation of the partition obtained by the application of the Label Propagaion algorithm.}
        \label{labelprop}
    \end{table}

% section label_propagation (end)

\section{Louvain} % (fold)
\label{sec:louvain}
    The application of the \textbf{Louvain} algorithm returns the best partition among all the partitions returned
    by the other algorithms. The results of its application are represented in Table \ref{louvain}.

    \begin{table}[H]
        \centering
        \begin{subtable}{\textwidth}
            \resizebox{\textwidth}{!}{
                \begin{tabular}{| c | c | c | c | c | c | c | c | c |}
                    \hline
                    Communities & Biggest & Smallest & Modularity & Conductance & IED & AND &  Tags & Langs \\
                    \hline
                    731 & 34 & 1 & 0.75 & 0.070 & 0.14 & 1.62 & cambridgeanalytica, deletefacebook, privacy,
                    zuckerberg, facebook, facebookgate & English, French, Deutsch, Arabic \\
                    \hline
                \end{tabular}
            }
        \end{subtable}
        \caption{Evaluation of the partition obtained by the application of the Louvain algorithm.}
        \label{louvain}
    \end{table}

% section louvain (end)

\section{Girvan-Newman} % (fold)
\label{sec:girvan_newman}
    For the \textbf{Girvan-Newman} algorithm, we've decided to record the results of $5$ iterations over the sample
    network. The results are represented in Table \ref{girvan_newman}. As you can see, the first iteration
    returns a very poor partition, with a low modularity score, due to the fact that the edge with the highest
    betweenness centrality in the starting sample network doesn't provide a good grade of separation among the nodes
    of the network. With the second iteration, and the ones after that, there is a consistent improvement either in
    the modularity score and in the other measures.
    \begin{table}[H]
        \centering
        \begin{subtable}{\textwidth}
            \resizebox{\textwidth}{!}{
                \begin{tabular}{| c | c | c | c | c | c | c | c | c | c |}
                    \hline
                    Iteration & Communities & Biggest & Smallest & Modularity & Conductance & IED & AND &  Tags &
                    Langs \\
                    \hline
                    1 & 720 & 234 & 1 & 0.19 & 0.0016 & 0.21 & 1.24 & cambridgeanalytica, deletefacebook, privacy,
                    zuckerberg, facebook, facebookgate & English, French, Deutsch, Arabic, Spanish, Italian,
                    Portoguese, Hindi, Finnish, Hungarian \\
                    \hline
                    2 & 721 & 117 & 1 & 0.55 & 0.0077 & 0.20 & 1.32 & zuckerberg, cambridgeanalytica,
                    deletefacebook, facebook, facebookgate & English, French, Deutsch, Hindi, Finnish, Spanish \\
                    \hline
                    3 & 722 & 117 & 1 & 0.59 & 0.016 & 0.19 & 1.36 & cambridgeanalytica, deletefacebook, privacy,
                    zuckerberg, facebook, facebookgate & English, French, Deutsch, Arabic, Italian, Portoguese,
                    Hungarian \\
                    \hline
                    4 & 723 & 109 & 1 & 0.61 & 0.018 & 0.18 & 1.42 & cambridgeanalytica, deletefacebook, privacy,
                    zuckerberg, facebook, facebookgate & English, French, Deutsch, Arabic, Italian, Portoguese,
                    Hungarian \\
                    \hline
                    5 & 724 & 96 & 1 & 0.65 & 0.023 & 0.18 & 1.49 & zuckerberg, cambridgeanalytica, deletefacebook,
                    facebook, facebookgate & English, French, Deutsch, Italian, Hindi \\
                    \hline
                \end{tabular}
            }
        \end{subtable}
        \caption{Evaluation of the partition obtained by the application of the Girvan-Newman algorithm.}
        \label{girvan_newman}
    \end{table}

% section girvan_newman (end)

\section{Demon} % (fold)
\label{sec:demon}

\begin{table}[H]
    \centering
    \begin{subtable}{\textwidth}
        \resizebox{\textwidth}{!}{
            \begin{tabular}{| c | c | c | c | c | c | c | c | c | c |}
            \hline
            Epsilon & Communities & Biggest & Smallest & Modularity & Conductance & IED & AND &  Tags & Langs \\
            \hline
            \end{tabular}
        }
    \end{subtable}
    \caption{Evaluation of the partition obtained by the application of the Demon algorithm.}
    \label{demon}
\end{table}

% section demon (end)


    \include{chaps/robustness/robustness}

    \chapter{Summary}




\printbibliography[title={References}]


\end{document}
